\section{Performance}

\subsection{SD Card Performance}
\label{sec:performance-sd-card}

As mentioned in section \ref{sec:avr-spi-issues}, reading data from an
SD card over SPI turned out to be the biggest bottleneck for throughput
in our system. In this section, we will document the optimizations we
made to improve the SD card's read performance.

Table \ref{tab:spi-optimizations-table} shows the gradually improving
performance as we optimized the code. This is also shown in figure
\ref{fig:spi-optimizations-plot}. Each of the optimizations are
explained in greater detail below.

\TODO{trenger vel kanskje ikke både figur og tabell?}
\TODO{lage et finere plot?}

\begin{description}
	\item[Removed status checking + function inlining] \hfill \\
		The spi\_send function busy-waited on the TXEMPTY flag for each
		byte it transmitted. In our case of alternating sends and reads,
		we do not have to check whether the SPI send register is empty
		every time. Inlining spi\_read and spi\_write also helped,
		although -O1, -O2 or -O3 would have done this for us.
	\item[Less SPI status register polling] \hfill \\
		In spi\_read, two conditions are busy-waited on before
		returning: whether the send register is empty, and whether the
		receive buffer is full yet. We only have to check the latter of
		these conditions.
	\item[Increased clock rate] \hfill \\
		Our CPU and main bus is driven by an external 12MHz crystal
		oscillator and the SPI clock frequency is calculated by dividing
		this by an integer number. We kept this divisor to be either 1
		or 2 (so SPI ran at either the same or half the speed of the
		CPU). Increasing to clock rate from 12MHz to 78 MHz boosted the
		speed from 157kB/s to 456 kB/s.
	\item[Compiler optimizations] \hfill \\
		Turning on compiler optimizations improved increase performance
		by another 14\%. There was no noticeable difference in
		performance from -O1, -O2 or -O3. However, compiler
		optimizations had a more profound effect on the overall system
		performance. We did not benchmark this extensively, but it was
		probably caused by optimizations the compiler made to the \ac{LENA}
		communication code.
	\item[Bypass FAT and use multiple block reads] \hfill \\
		We patched the framework code to support multiple block reads.
		Apparently, this was not something the FAT driver was able to
		take full advantage of out of the box. It was easier to simply
		bypass the file system than to try to fix it, so that is what we
		did.
		
		By reading the blocks directly from the SD card, we could read
		all 150 blocks of a picture in one function call. Limiting the
		file system to the first half of the SD card and add metadata
		files containing the block offsets into the second half of the
		card was easy. This more than doubled our performance, from 515
		kB/s to 1172 kB/s.
\end{description}

\TODO{This must be rewritten}
After implementing these optimizations, the \ac{SCU} was capable of pushing out around
12 FPS to the \ac{LENA}, around 920 kB/s. As we managed to reach our goal of 10
FPS, we did not optimize it any further. However, using the \ac{PDCA}, we could
have effectively eliminated the 25\% time spent sending to the \ac{LENA} while
the \ac{SD} card waited. Thus, giving as a frame rate of around 16
FPS. 
\begin{table}[h!]
\centering
\begin{tabular}{c|c|c}
	\textbf{Milestone} & \textbf{Description} & \textbf{Performance} \\
	\hline
	1 & Baseline before optimization & 74.87 kB/s \\
	\hline
	2 & Less status checking in spi\_send + inlining & 137 kB/s \\
	\hline
	3 & Less SPI status register polling & 156.59 kB/s \\
	\hline
	4 & Clock rate of 20MHz, 10MHz SPI & 214.4 kB/s \\
	\hline
	5 & Clock rate of 20MHz, 20MHz SPI & 243.0 kB/s \\
	\hline
	6 & Clock rate of 40MHz, 20MHz SPI & 327.8 kB/s \\
	\hline
	7 & Clock rate of 48MHz, 24MHz SPI & 369.5 kB/s \\
	\hline
	8 & Clock rate of 60MHz, 30MHz SPI & 423.4 kB/s \\
	\hline
	9 & Clock rate of 72MHz, 36MHz SPI & 432.7 kB/s \\
	\hline
	10 & Clock rate of 78MHz, 39MHz SPI & 455.8 kB/s \\
	\hline
	11 & Turn on -O1 & 521.7 kB/s \\
	\hline
	12 & Turn on -O2 & 523.4 kB/s \\
	\hline
	13 & Turn on -O3 & 515.3 kB/s \\
	\hline
	14 & Bypass FAT and use multiple block reads & 1171.9 kB/s \\
\end{tabular}
\caption[SPI Read Performance]{The read performance over SPI at various
milestones.}
\label{tab:spi-optimizations-table}
\end{table}

\begin{figure}[h!]
\centering
\includegraphics[width=\textwidth]{fig/avr/spi-optimizations-plot.pdf}
\caption[SPI Optimizations Plot]{The results from table
\ref{tab:spi-optimizations-table} plotted, which milestones along the
x-axis and read performance along the y-axis.}
\label{fig:spi-optimizations-plot}
\end{figure}


\begin{comment}
		In this section, there will be data as specified in the disposition. We will run
a program which requires data from neighbors, and see how long time the
different programs take based on the amount of cores. We will then explain
eventual results | for instance would the results most likely point out that we
have low latency, and that it will take a very short amount of time to get data
from neighbors.

There will also be some fancy plots and tables here to show the decrease in time
and/or the increase in speedup.
\end{comment}
