\section{Introduction}
\TODO{write!}

Our AVR is a AT32UC3A0512 this was both an advantage and a disadvantage to us. This is the same AVR that have been used in previous years which allowed us to learn from their success and failures to a greater degree. At the same time it also made some problems for us especially in regards to performance on our SD card. 

When we first started working on the AVR we focused mostly on the hardware mappings as this was needed for the PCB to be delivered. We made sure to make it as similar as possible to testing cards avaliable to us as to make for easier testing of our code for this project. Having completed this let us look at the code and to define what we figured we would need in terms of functions and headers for our application. Granted this was not set in stone as much of the design was yet to be determined but it gave us a starting point to which we were able to work and improve upon.

We intended for the AVR to be the controlling unit in our system. The AVR is in charge of setting up memory (both data and instructions) for the FPGA to do work on. After the processing is done the FPGA will tell the AVR by setting an interrupt bit and as such letting the AVR know that the work has been completed. It's then up to the AVR to decide what to do next. 

The AVR also implements the user interface for the project. With the help of the VGA implementation on the FPGA it is able to display a menu to the user which allows for selection of a program and data set from the SD card connected to the AVR.
