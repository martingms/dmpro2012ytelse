\section{AVR hardware}

This section documents the hardware consideration we have to make in relation to the AVR in our project. Also included are spesifics in regards to what choices were made in regards to AVR spesific hardware.

The AVR had a number of considerations in mapping the pinout. First off we were wanting to have a number of functions that required specific pins on the AVR to mention some:

\begin{enumerate}
\item Serial port
\item SD card 
\item USB connectivity
\item EBI (avrMemory)
\end{enumerate}

We also wanted to have some visual input and debugging tools for the project (buttons and leds). These were also connected to the AVR, we did however not have to take that into account mapping pins as they can be fitted on any GPIO pin. This allowed us greater flexibility in terms of what had to go where and made it possible to fit the other interfaces on the AVR better.

Same goes for the VGA controller we had mapped up to the AVR. Seeing as this neither did require specific ports on the AVR we fit this in where there were leftover pins when the rest was mapped out.

As for the actual pinout of the AVR refer to the appendix.
\TODO Add appendix on this

\subsection{AVR crystal}

Previous years have chosen to have multiple crystals for the AVR. Our AVR has support for up to 3 crystals where 2 is high frequency crystals that can be used as external clocks for the AVR. The last crystal is a low frequency crystal that is needed if one intends to make use of power saving functions in the AVR or are in need of real time measurement on the AVR.

As we were to optimize for performance we tried to use the most of the given resources. This including saving pins wherever we could, and as such we have gone with only one crystal. This crystal is connected to port 0 on the AVR and has a socket. We ordered a number of crystals in order to make sure that we could adapt to most any requirements that may or may not have been clear to us at the time of the design.

\subsection{SD card}

We also needed a storage medium, since previous years have had good results with SD card this was what we chose as well. However working on the design we encountered a few problems in regards to the SD card, mostly in regards to performance more about this refer to the SD card implementation section.

\subsection{Serial port}

The serial feature was implemented as a way of debugging. This is connected to USART0 on the AVR and was not a crucial part of the design but rather a way for us to communicate and get quantifyable data from the AVR. We decided having serial access would be handy as there is only so much data you can get out of 8 blinking leds. The serial implementation had some issues in regards to our mapping of the lines to the port, this was partly tried to solve by remapping the lines over the header.

\subsection{USB connectivity}

Our AVR also supported USB. We decided to map up the USB even though we were doubtful if it would have much use for it due to both having SD card and serial for debugging purposes. This feature therefore got low priority and was not implemented in the final design due to time constraints.
 
\subsection{EBI (avrMemory)}

Early in the design process we came to doubt wheter the SD card would perform as well as we needed it to. In order to try to safeguard us from total failure in a case where this would come true (which it did) we gave the AVR a seperate memory in order to be able to have a buffer for data comming from the SD card or from the FPGA. The AVR itself don't have enough memory for even a single fram of video and as such this was after some discussion deemed a worthwhile use of ports at the expense of some I/O lines to the FPGA.

\subsection{VGA controller}

Questioning wheter or not we would be able to implement a vga controller on the FPGA we thought it wise to have a backup to get picture on a screen should this fail. The VGA controller is a prebuilt type that have been previously used in projects with some success. We added this as a backup where we would pipe data back to the AVR for display on screen should the FPGA VGA controller fail.


