\chapter {PCB}\label{ch:pcb}

\begin {flushright} {\slshape
    In which copper sheets are etched onto a red carpet\\
    connecting the different components together\\
    realizing their synergy and power
}
\end {flushright}
...Abstract...

Design choices: \\
Memory: \\
Since we decided to have separate instruction/data-memories, as well as VGA-ram, and
a dedicated memory for the AVR, we ended up with a total of 5 RAM-chips. The requirements
on the various chips did differ a bit though, we needed 8-bit words for all data, but
24 bit for instructions. Since 24-bit memory was out of production, and 32-bit memory
was too expensive, we opted for using two 16-bit chips, with their address-lines connected
together, and 8 ignored I/O-pins, thus in effect creating a 24-bit memory. \\

Since we were unsure about how constant we could get our data-streams, an additional
AVR-memory was added. (TODO: This sentence needs to fit in with the above, AND, the
below notion of fallbacks)

VGA: \\
While we had a goal of creating our own VGA-controller in the FPGA, we decided to have
some fallback-option, thus, in addition to the neccessary pins and connectors for that
solution, we also added the necessary connectors for the VGA-connector that Festina Lente
used last year.

Communication: \\
We planned on using the SD-Card-reader as our main source of data/instructions, however
in the same vein as the design choices for the VGA, we opted to also have a fallback-solution
here, thus we also added USB and RS232 as fallback-solutions for getting data/communicating
with the computer.

\section {Schematics}

We decided to design the entire PCB in one schematic, because having read
the Festina Lente report, we noticed that they mention that using multiple schematics
might give some interesting issues (which we never got any further into,
as we never tried to use more than one schematic).

A major downside of this approach was the fact that this completely serialized our
work on the Schematic, since we couldn't make concurrent changes to our single document.
This limited the paralellism of our work-flow to one person working on the schematic,
and the other people in the PCB-group doing whatever could be done without touching the schematic.
(Looking up components, making footprints etc). Since the schematic was the biggest amount
of work, this produced a major bottle-neck for the schematic work. HOWEVER, having the overall
control of the entire thing in one document did help to smooth out quite a few issues we met
along the road. (TODO: Fill in a bit of details about pros for one-document vs multi).

The overall layout of the schematic is logically grouped \"geographically\", to
allow for easy reading of the schematic.

\subsection {Buses/Wirelabels}
(TODO: Noe om issues vi fikk med å navngi busser)

TODO: Overordnet forklaring av hvordan

REMOVE THIS WHEN DONE
\section {Dispostition}
- requirements avr, fpga
- fpga ready footprint
- how did we get avr footprint?
- layers
- all in one schematic, no parallelization
- redundance, examples
- simple, examples (toplevel, chapter 1?)
- headers
- footprints and schematics
- tests, during soldering,

for every part (schematic section, footprint) etc
    - references, datasheets
    - differences from last year
    - diffictulties
    - pictures, in text of refer to figure id something something
    - schematic section/part

    - some sentences about using the macaos tool
    - time usage
    - week routing, compare to other group
    - one memory chis vs 5
    - soldering insights
    - two vgas (top level design?)
    - memory choice, why, explain 2*16-8 bit (is this PCB or "system"?)
    - autorouting + manual via removal
    - find physical proximity
    - accidentally forgot one pin in schematic during reordering
    - could have used other auto-route settings
    - think physical prox when throughout all the planning, at least pin "budget"
    - Lena on top overlay, how, edge detection ++, ALtium Delphi-script
    - problems with oscillator, (despite not mirrored footprint like last year)

    - buses, net-label, unique names

    - memory chip design, why so many, why 2 instr mem chips?
    - 24 bit chips out of production
    - 32 bits too expensive


    part list:
    - capacitors, resistors (why, which)
    - 1206, copy from last years success
    - avr
    - voltage regulator area, cribbing with Gunnar's and Magnus' blessing
    - power plane
    - usb
    - serial
    - sd card
    - avr memory, vga memory, data memory,
    - avr vga, fpga vga
    - group communication?

    - via sizes, 0.71mm?

    lots and lots of referces and source citations


    Appendiks:
    - Pin "usage"
    - Schemetics, overview + sections (mspaint-PNG? vector graphics?)
    - ask other group about vector graphics (they used for logo)
        - pictures of PCB
        - footprints




\section {Power supply}

After talking to Gunnar and Magnus, we ended up with cribbing the power supply from
Festina Lente completely, with no changes, as Gunnar said the power supply had
evolved year by year, and would thus be a better solution than trying to
create one from scratch ourselves (reducing the possibility for additional
problems).

\section {Power Plane}

\begin{figure}[h]
  \centering
  \includegraphics[width=0.8\textwidth]{fig/pcb/power_planes.PNG}
  \caption{The Power Planes}
  \label{fig:power_planes}
\end{figure}


Since our power supply was exactly the same as Festina Lente's we also ended up
with similar power planes, as seen in Figure \ref{fig:power_planes}; 12V (dark blue), 5V (teal), 
3.3V (red), 2.5V (gray) and 1.2V (green). The 5V was only used for the external 
\ac{VGA} module. 2.5V and 1.2V were split across the \ac{FPGA}.

The rest of the board got 3.3V. The entire power plane was put in internal layer
1, with a ground layer in internal layer 2.

\section {Footprints}
Some of the components we chose didn't have footprints readily available,
which meant we either had to look for them on the internet, or create some
ourselves.

(TODO: Reference data-sheets for the foot-prints)

\subsection {We made the following footprints ourselves}
\begin {itemize}
\item SD-card
\item VGA-plugg
\item Memory (TSOP54/TSOP44)
\end {itemize}

\subsection {Footprints from Festina Lente:}
As Festina Lente used some components that we also ended up using, we decided,
after talking to Gunnar, to use their Footprints (as they were known good) for these
components:
\begin {itemize}
\item Button
\item USB
\item PowerConnector
\item Crystal
\item Oscillator
The oscillator had one issue last year, namely that the footprint was mirrored,
noting this from Festina Lente's report, we read through the datasheet (REFERENCE),
and remapped the pin-numbering on the foot-print before using this footprint.
\end {itemize}
Molex?
VGA-module?

\subsection {Stuff about AVR-footprint}
\subsection {Stuff about FPGA-footprint}



\section {Routing}

We spent the entire final week before production working on the routing,
the reason for the time span needed for this was a result of a couple of
things.

(TODO: Expand the following into text)
Why we used so much time: (Or, what the other group did right, and we did wrong)
\begin {itemize}
\item We were the first group to start routing, and thus got to fall into all the gotcha-traps that
exist for routing, with no-one to walk into them ahead of us.
\item We didn't think about physical proximity when laying out the pins
\item We had 5 memory chips, with all the extra connections that implies.
\item We didn't set limitations until a few days into the week.
\end {itemize}

Initially we attempted auto-routing, which took the better part of half an hour,
even before the constraints were set, this showed us quite a few issues that needed
to be handled (and even more so when we finally got the constraints in).

TODO: Fill in about the types of errors/warnings we got.
\begin{itemize}
\item Constraints violations set wrong
\item Power-plane net-labeled wrong.
\item Clearance constraints violated by Altium
\item Short-circuiting vias.
\item Overlapping vias.
\item In general, the auto-routing started to produce violations 
\item more...
\end{itemize}
TODO: Silk-over-silk in USB/Lenna, errors we could ignore.
TODO: Scaling of board. (Should go elsewhere or not at all?)
Initially we started with a boardsize chosen more or less at random, just picking something
that seemed to fit the components comfortably with room to spare, after laying out the components
on the board, we noticed that the board had quite a bit of room left over, as this would be a waste
of resources, we scaled the board down in size, by moving the keep-out-borders inwards until the wasted
space was removed, and then using the "scale-to-fit-components"-tool in Altium, with the Keepout-border
selected.

In the end we remapped some pins to increase physical proximity, and to untangle the
amount of crossing wires. Although care was taken during this process, we accidentally
happened to disconnect one pin in the schematic while reordering, we did notice this before
production though, and were able to correct the mistake by manually routing the connection.
Finally we did some manual routing to reduce the amount of unneccessary vias. \\

\subsection {Lenna and Text}
In the top right corner of our board, we put the "Lenna"-picture, this image
was produced by doing an edge-detection-pass on the image, then putting it through
the image-converter-plugin in Altium: \url{http://wiki.altium.com/display/ADOH/How+to+import+a+graphic+onto+the+PCB+overlay}

We also put our names, and the name of the project on the board, this was simply done
with the Text-label tool in Altium's PCB-Designer. This didn't allow for norwegian letters,
but this was fixed easily by adding lines and circles manually (as well as moving a's and e's together
to form æ).

\section {Macaos}

(TODO: This section is intended to provide a few insights about the
ins and outs of using Macaos for creating production-files, some of
what ends up here MAY fit better in an Appendix, but we feel it might
be usefull to note down what we learned for those that come after us)

\section {The PCB}
\begin{figure}[h]
  \centering
  \includegraphics[width=0.8\textwidth]{fig/pcb/pcbwithoutcomp.jpg}
  \caption{The PCB without Components}
  \label{fig:pcb}
\end{figure}
When the PCB finally arrived from production we started soldering, this proved to be a bit
more difficult than originally anticipated. First, we tested whether there were short circuit in the board itself. We used multimeter to test that there was no current that goes from different layer. Then, we started to solder with the power supply, one power plane at time, testing the board for short circuit after each iteration. The table 5.1 shows the observerd voltage from each plane. 
\begin{table}[h]
  \centering
  \begin{tabularx}{\textwidth}{l l l l}\toprule
    \thx{Test} & \thx{Result} & \thx{Passed} 
    \\ 
	 \midrule
    Power supply 12.0V               &Measured 12.045  & OK  \\	
    Power supply 5.0V               &Measured 4.995  & OK  \\
    Power supply 3.3V                   & Measured 3.286 & OK  \\
    Power supply 2.5V                 & Measured 2.510 & OK \\
    Power supply 1.2V            & Measured 1.240 & OK  \\
    
    \bottomrule
  \end{tabularx}
  \caption{Results of power supply}
  \label{fig:pcb}
\end{table}


Then we started with AVR, getting the AVR in place was particularly difficult, but after asked Tufte we lent a glue stick to put some glue on it to get more friction, thus keeping it in place while we soldered.

After about a days work we managed to completely destroy a pin on the AVR, therefore we had to start from scratch on a new board. On this new board we started with the AVR and FPGA, as they are the hardest components to solder. After each side on the AVR and FPGA, we tested for short circuits. As none were found, we moved on to the power supply, as well as JTAGs , FLASH and LEDs. In order to check the PCB was working, the AVR and FPGA groups tested the PCB board without the capacitors, after both groups tested that they could connect to the AVR and FPGA respectively, we began to solder the rest part of the PCB. 
\begin{figure}[h]
  \centering
  \includegraphics[width=0.8\textwidth]{fig/pcb/pcbwithcomp.jpg}
  \caption{The PCB With All The Components}
  \label{fig:pcb}
\end{figure}

\section {Problems and Workarounds}
\subsection{Routing}
Routing was the first problem we confronted during the designing the \ac{PCB}
board, we had to manually route the design, because we got some warnings due to
the clearance of the board. We used the auto-routing function provided by
Altium, yet as it did not privode satisfying results, we tried to change some
parameters to make them better, without success. Finally, we realized that we
had to route a few wires manually. This was mostly motivated by our desire to
not have vias as close to the pins as Altium placed them. Also, as to cut down
the bugdet, we removed some unnecessary vias by wiring multiple grounds and
powers together, where the design allowed us to. In this way, we got quite a few
less vias.
\subsection{Soldering}
Soldering was the bigest problem that we faced, none of the \ac{PCB} group
member had any experiences with it beforehand. We managed to destroy our first
board, as we used too much tin on one of the AVR pins, and ruined them by using
the tin removal unit wrongly.

The next try, however, proved to be successful, but there still remained some
minor problems. Problems we found were lack of tin on the oscillator, as well as
a few of the \ac{FPGA} pins. They were, as opposed to our problem with the first
board, easily fixed, once discoverred.



