\section{SIMD-nodes}

\TODO{This is very detailed and more of a documentation, rather than an
  explanation of how the SIMD nodes work. This section should be rewritten
  entirely (as per issue \#3).}

All SIMD nodes shares the same instruction set and execute instructions in
parallel. {\tt Word} size is 8 bit. Each SIMD node is fully equipped with
registers, an aritmetic logic unit (ALU), message passing and instruction
handling through the SIMD Node instruction set, which is explained within this
section.

The schematic of a SIMD node is shown in figure
\ref{fig:fpga-simd-arch}. \TODO{Talk around the figure}

\begin{figure}[h]
  \centering
  \includegraphics[width=\linewidth,clip,trim=0 11cm 0 0]
                  {fig/fpga/fpga-simd-arch.pdf}
  \caption{LENA SIMD architecture}
  \label{fig:fpga-simd-arch}
\end{figure}


\subsection{Components}
In order to the keep the signals to a minimum, the SIMD node is nicely divided
into separate components which makes up the datapath for the node.

\subsection{Instruction Decoder}
The instruction decoder is the control component of a node. It takes the opcode
of the instruction and sets control signals for all the other components in the
node.

\subsection{I/O Controller}

\subsection{Register Bank}

\subsection{ALU}

\subsection{S Register}
The source data register, S REG for short, is a special purpose register within
the SIMD node. S REG holds the next source data for the node. It is partly
controlled by the instruction set for the node and partly controlled by a
special {\tt step} signal from the DMA.

This register also has the capability to receive data from the left node,
through the {\tt s\_in}-bus, and passing it along to node on the right through
the {\tt s\_out}-bus when instructed by the {\tt step} signal. This allows a
simultaneous data transfer while the node is busy processing.

Rising the value on the {\tt swap} control signal will write the result from the
ALU to the S\_REG and copy the data from the S\_REG out to the {\tt s\_new} bus,
ultimately writing this data to the register bank.

\subsection{State Register}

\subsection{Registers}
Each SIMD node have $2^4 = 16$ general purpose registers. 4 of these are
available for general storage when executing instructions. The remaining 2
registers are the special purpose registers {\tt \$zero} and {\tt \$state}.

\begin{table}[h]
  \centering
  \begin{tabularx}{\linewidth}{XXXXXXXXX}\toprule
    R0 & R1 & R2 & R3 & R4 & R5 & R6 & R7 \\ \midrule
    \tt \$zero & \tt \$r1 & \tt \$r2 & \tt \$r3 & \tt \$r4 & \tt \$r5 &
    \tt \$r6 & \tt \$r7\\ \bottomrule
  \end{tabularx}
  \begin{tabularx}{\linewidth}{XXXXXXXX}
    R8 & R9 & R10 & R11 & R12 & R13 & R14 & R15 \\ \midrule
    \tt \$r8 & \tt \$r9 & \tt \$r10 & \tt \$r11 & \tt \$r12 & \tt \$r13 &
    \tt \$r14 & \tt \$state\\ \bottomrule
  \end{tabularx}
  \caption{Registers in the SIMD nodes}
  \label{tab:simd-registers}
\end{table}


\subsection{State-register}

\subsection{Zero-register}

\subsection{BRAM and SRAM}
BRAM or SRAM is not available from the SIMD node.

\subsection{Instruction Set}
SIMD nodes operates on a {\tt 24 bit} instruction set divided in 2 main
formats. Arithmetic instructions (R, I and S) and message passing instructions
(M-send, M-store and M-forward).

\subsection{Ideology}

\subsection[R-Format]{R-Format (OP = 000)}
Arithmetic register functions instructions

\begin{table}[h]
  \centering
  \begin{tabular}{cccccccc}\toprule
    \thx{ctrl} & \thx{mask} & \thx{op} & \thx{rs} & \thx{rt} & \thx{rd} &
    \thx{n/a} & \thx{fn} \\ \midrule
    1 bit & 1 bit & 3 bit & 4 bit & 4 bit & 4 bit & 4 bit & 3 bit
    \\ \bottomrule
  \end{tabular}
  \caption{Arithmetic register function instructions}
  \label{tab:ar-re-fu-in}
\end{table}


\begin{itemize}
\item {\tt ctrl} must be set to 0 in order to be executed on the SIMD node.
\item {\tt mask} is set to 1 to selectively enable the node when executing
  conditional branches.
\item {\tt op} the instruction opcode.
\item {\tt rs} write data register address.
\item {\tt rt} read data 1 register address.
\item {\tt rd} read data 2 register address.
\item {\tt n/a} not assigned for R-instructions.
\item {\tt fn} the artimetic operation to perform.
\end{itemize}

\begin{table}[h]\small
  \centering
  \begin{tabularx}{1.03\textwidth}{lccX}\toprule
    \thx{name} & \thx{fn} & \thx{assembly code} & \thx{binary representation}
    \\ \midrule
    \thx{add} & \tt 000 & \tt add \$rs, \$rt, \$rd &
    \tt 0 m 000 ssss tttt dddd ---- 000\\
    \thx{sub} & \tt 001 & \tt sub \$rs, \$rt, \$rd &
    \tt 0 m 000 ssss tttt dddd ---- 001\\
    \thx{slt} & \tt 010 & \tt slt \$rs, \$rt, \$rd &
    \tt 0 m 000 ssss tttt dddd ---- 010\\
    \thx{and} & \tt 011 & \tt and \$rs, \$rt, \$rd &
    \tt 0 m 000 ssss tttt dddd ---- 011\\
    \thx{or}  & \tt 100 & \tt or~ \$rs, \$rt, \$rd &
    \tt 0 m 000 ssss tttt dddd ---- 100\\
    \thx{eq}  & \tt 101 & \tt eq~ \$rs, \$rt, \$rd &
    \tt 0 m 000 ssss tttt dddd ---- 101\\
    \thx{sll} & \tt 110 & \tt sll \$rs, \$rt, \$rd &
    \tt 0 m 000 ssss tttt ---- ---- 110\\
    \thx{srl} & \tt 111 & \tt srl \$rs, \$rt, \$rd &
    \tt 0 m 000 ssss tttt ---- ---- 111\\ \bottomrule
  \end{tabularx}
  \caption{List of R instructions}
  \label{tab:r-instructions}
\end{table}


\subsection[I-Format]{I-Format (OP = 001)}
Immediate functions using constants.

\begin{table}[h]
  \centering
  \begin{tabular}{ccccccc}\toprule
    \thx{ctrl} & \thx{mask} & \thx{op} & \thx{rs} & \thx{rt} & \thx{const} &
    \thx{fn} \\ \midrule
    1 bit & 1 bit & 3 bit & 4 bit & 4 bit & 8 bit & 3 bit
    \\ \bottomrule
  \end{tabular}
  \caption{Immediate functions using constants}
  \label{tab:immediate-fn-const}
\end{table}


\begin{itemize}
\item {\tt ctrl} must be set to 0 in order to be executed on the SIMD node.
\item {\tt mask} is set to 1 for selectivly enabling the node when executing
  conditional branches.
\item {\tt op} the instruction opcode.
\item {\tt rs} write data register address.
\item {\tt rt} read data 1 register address.
\item {\tt const} constant value | immediate.
\item {\tt fn} the artimetic operation to perform.
\end{itemize}

\begin{table}[h]\small
  \centering
  \begin{tabularx}{1.03\linewidth}{lccX}\toprule
    \thx{name} & \thx{fn} & \thx{assembly code} & \thx{binary representation}
    \\ \midrule
    \thx{add} & \tt 000 & \tt addi \$rs, \$rt, \$rd &
    \tt 0 m 001 ssss tttt cccccccc 000\\
    \thx{sub} & \tt 001 & \tt subi \$rs, \$rt, \$rd &
    \tt 0 m 001 ssss tttt cccccccc 001\\
    \thx{slt} & \tt 010 & \tt slti \$rs, \$rt, \$rd &
    \tt 0 m 001 ssss tttt cccccccc 010\\
    \thx{and} & \tt 011 & \tt andi \$rs, \$rt, \$rd &
    \tt 0 m 001 ssss tttt cccccccc 011\\
    \thx{or}  & \tt 100 & \tt ori~ \$rs, \$rt, \$rd &
    \tt 0 m 001 ssss tttt cccccccc 100\\
    \thx{eq}  & \tt 101 & \tt eqi~ \$rs, \$rt, \$rd &
    \tt 0 m 001 ssss tttt cccccccc 101\\
    \thx{sll} & \tt 110 & \tt slli \$rs, \$rt, \$rd &
    \tt 0 m 001 ssss tttt -------- 110\\
    \thx{srl} & \tt 111 & \tt srli \$rs, \$rt, \$rd &
    \tt 0 m 001 ssss tttt -------- 111\\ \bottomrule
  \end{tabularx}
  \caption{List of I instructions}
  \label{tab:i-instructions}
\end{table}


\subsection[S-Format]{S-Format (OP = 010)}
Swap source data register with processed data and store new source data in
register.

\begin{table}[h]
  \centering
  \begin{tabular}{cccccccc}\toprule
    \thx{ctrl} & \thx{mask} & \thx{op} & \thx{rs} & \thx{rt} & \thx{rd} &
    \thx{n/a} & \thx{fn} \\ \midrule
    1 bit & 1 bit & 3 bit & 4 bit & 4 bit & 4 bit & 4 bit & 3 bit \\ \bottomrule
  \end{tabular}
  \caption{S format instructions}
  \label{tab:s-fmt-instr}
\end{table}


\begin{itemize}
\item {\tt ctrl} must be set to 0 in order to be executed on the SIMD node.
\item {\tt mask} set to 1 for selectivly enabling the node when executing
  conditional branches.
\item {\tt op} the instruction opcode.
\item {\tt rs} new source data register address.
\item {\tt rt} old source data register adddress.
\item {\tt rd} must be 000000.
\item {\tt n/a} not assigned for S-instructions.
\item {\tt fn} must be 000.
\end{itemize}

\begin{table}[h]\small
  \centering
  \begin{tabularx}{\textwidth}{lccX}\toprule
    \thx{name} & \thx{fn} & \thx{assembly code} & \thx{binary representation}
    \\ \midrule
    \thx{swap} & \tt 000 & \tt swap \$rs, \$rt &
    \tt 0 m 001 ssss tttt 0000 ---- 111\\ \bottomrule
  \end{tabularx}
  \caption{List of S instructions}
  \label{tab:s-instructions}
\end{table}



\subsection[M-Format]{M-Format (OP = 100, 101, 110)}
Send, recieve and forward data from and to neighbour nodes in all directions (north, south, east and west).

\begin{table}[h]
  \centering
  \begin{tabular}{cccccccc}\toprule
    \thx{ctrl} & \thx{mask} & \thx{op} & \thx{n} & \thx{s} & \thx{e} &
    \thx{w} & \thx{n/a} \\ \midrule
    1 bit & 1 bit & 3 bit & 4 bit & 4 bit & 4 bit & 4 bit & 3 bit \\ \bottomrule
  \end{tabular}
  \caption{M format instructions}
  \label{tab:m-fmt-instr}
\end{table}


\begin{itemize}
\item {\tt ctrl} must be set to 0 in order to be executed on the SIMD node.
\item {\tt mask} set to 1 for selectivly enabling the node when executing
  conditional branches.
\item {\tt op} the instruction opcode.
\item {\tt n} north data write or read register address.
\item {\tt s} south data write or read register address.
\item {\tt e} east data write or read register address.
\item {\tt w} west data write or read register address.
\item {\tt n/a} no ALU operations applicable.
\end{itemize}

\begin{table}[h]
  \begin{tabularx}{\textwidth}{lrcc}\toprule
    \thx{name} & \thx{OP} & \thx{assembly code}
    \\ \midrule
    \thx{send} & \tt 100 & \tt send~ \$r1, \$r2, \$r3, \$r4\\
    \thx{store} & \tt 101 & \tt store \$r1, \$r2, \$r3, \$r4\\
    \thx{store \& forward} & \tt 110 & \tt frwrd \$r1, \$r2, \$r3, \$r4
    \\ \bottomrule
  \end{tabularx}
  \begin{tabularx}{\textwidth}{lX}
    \thx{name} & \thx{binary representation} \\ \midrule
    \thx{send} & 
    \tt 0 m 100 nnnn ssss eeee wwww ---\\
    \thx{store} &
    \tt 0 m 101 nnnn ssss eeee wwww ---\\
    \thx{store \& forward} & 
    \tt 0 m 110 nnnn ssss eeee wwww ---\\ \bottomrule
  \end{tabularx}    
  \caption{List of M-instructions}
  \label{tab:m-instructions}
\end{table}


In figure \ref{tab:m-instructions} we see the different \ldots. \thx{store \&
  forward} forwards according to the 4-way data exchange patterns (N
$\rightarrow$ E, E $\rightarrow$ S, S $\rightarrow$ W, W $\rightarrow$ N).

{\tt PRO TIP} \TODO{Better avoid calling it PRO TIP in the report}
M-instruction, especially send and forwards, must be issued with mask bit set to
0 in order to get data from all the nodes and not only those withich are
execuriting withing a conditional branch (status = 1).

\begin{figure}[h]
  \centering
  \includegraphics[width=\linewidth,clip,trim=0 18cm 0 0]
                  {fig/fpga/fpga-simd-datacom.pdf}
  \caption{Four-way communication in the LENA SIMD array.}
  \label{fig:fpga-simd-datacom}
\end{figure}
\TODO{Split up into 3 figures instead?}


\subsection{Branching}
Since all nodes run the same instructions, both parts of a branch must be
executed. Nodes are setting the {\tt state} to 1 in order to indicate that they
are executing within that part of the branch.

\begin{table}[h]
  \centering
  \begin{tabularx}{\textwidth}{rlcX}\toprule
    \thxc{step} & \thxc{instruction} & \thxc{state} & \thxc{description} \\
    \midrule
    0 & \tt // initial state & 0 & \\
    1 & \tt eq \$state \$r1, \$r2 & 1
    & Set state to 1 if branch should be taken for the node.\\
    \ldots & \tt // branch taken & 1 &
    Instructions for when the branch is taken.\\
    2 & \tt eq \$state, \$state, \$zero & 0 & Negate the state.\\
    \ldots & \tt // branch not taken & 0 & Instructions for when the branch is
    not taken.\\ \bottomrule
  \end{tabularx}
  \caption{Single level branching}
  \label{tab:single-level-branching}
\end{table}


\subsection{Multilevel branching}
Since state register is 8 it is possible to have up-to 8 nested branches by
shifting the current state to the left and adding the new state to the
end. Below \TODO{Refer to table, not relative to where it is placed} are the
instructions for performing a multilevel-branch.

\begin{listing}[h]
  \usemintedstyle{soderberg}
  \centering
  \begin{lenacode}
    # Initial state.
    node sll R3 STATE      # Save current register
                           # by shifting left
    node eq R4 R1 R2       # Calculate if branch is taken
                           # by the node
    node add STATE R4 R3   # Set the new state
      # Instruction for when the 
      # condition is satisfied
    node andi R3 STATE 254 # Save old state
                           # (254 = 1111 1110)
    node andi R4 STATE 1   # Save new state
                           # (001 = 0000 0001)
    node eq R4 R4 ZERO     # Negate current state
    node add STATE R4 R3   # Set new state
      # Instruction for when the 
      # condition is not satisfied
    node srl STATE STATE   # Revert to old state

    # (code after branching would follow here)
  \end{lenacode}
  \caption[Multilevel branching]{Multilevel branching in SIMD nodes}
  \label{lst:multilevel-branching}
\end{listing}

\CHECK{Could any of these tables be figures instead?}
