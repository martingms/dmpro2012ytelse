\section{Requirements Specification}

\begin{table}[h]
  \centering
  \begin{tabularx}{\textwidth}{l X}\toprule
    \thxc{Name} & \thxc{Description}\\ \midrule
    {\sc NFR1} & The machine must use a Xilinx Spartan 3 XC3S500E PQG208 FPGA\\
    {\sc NFR2} & The machine must use one AVR32 UC3A microcontroller\\
    {\sc NFR3} & The budget of $\sim$ 10 000 NOK must cover all \ac{PCB} and
    component costs\\
    {\sc NFR4} & The image processor should consist of multiple cores arranged 
    in a matrix\\
    {\sc NFR5} & The machine should be optimized for performance\\
    \bottomrule
  \end{tabularx}
  \caption[Non-functional requirements]{The non-functional requirements}
  \label{fig:nonfunc-req}
\end{table}


Creating a computer without specifying functional requirements would make it
difficult to both see how well we are progressing and to figure out if we've
actually reached our goal in the end. Non-functional requirements are important
to realize the computer: Both price and components were set by the assignment as
non-functional requirements. As such, both functional and non-functional
requirements were written down to give ourselves clear goals and
specifications. The non-functional requirements are shown in table
\ref{fig:nonfunc-req}. 

\begin{table}[h]
  \centering
  \begin{tabularx}{\textwidth}{c X c}\toprule
    \thx{Name} & \thxc{Description} & \thx{Priority}\\ \midrule
    {\sc FR1} & \FRI   & {\sc High}   \\
    {\sc FR2} & \FRII  & {\sc High}   \\
    {\sc FR3} & \FRIII & {\sc High}   \\
    {\sc FR4} & \FRIV  & {\sc Medium} \\
    {\sc FR5} & \FRV   & {\sc Medium} \\
    {\sc FR6} & \FRVI  & {\sc Low}    \\
    {\sc FR7} & \FRVII & {\sc Low}    \\ 
    \bottomrule
  \end{tabularx}
  \caption[Functional requirements]{The functional requirements}
  \label{fig:func-req}
\end{table}
\TODO{Why do we not have simd as a requirement? Why not parallelism?}


Table \ref{fig:func-req}, which shows the functional requirements includes a
relative priority between the different requirements. This priority tells us
what we have focused on, as well as what is important in terms of success of the
computer. Clearly, focusing on performance, as specified in {\sc FR1}, is more
important than having developer tools for the machine ({\sc FR6}).
\CHECK{Ensure FR numbers are correct.}

{\sc FR1} ensure the focus on performance. {\sc FR2} ensures that we do not end
up with a system which is not generally programmable. This is important, as a
computer specialized for image processing has less usability than a generally
programmable computer. {\sc FR3} and {\sc FR7} makes it easier to show that the
computer is capable of processing images, whereas {\sc FR4 - FR6} makes it
easier to use, debug and create programs for the computer.
