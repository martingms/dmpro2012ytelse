\chapter{Results}\label{ch:res}

\begin{flushright}{\slshape
    If it's not on fire, it's a software problem.\\ \medskip
    --- Unknown}
\end{flushright}

When designing a system aimed for performance, one has to test the system and
compare it to other solutions to see whether we have gained any mentionable
speedup. In this section, As the assignment was to create an array-based
image processor, it would be interesting to see how our system performs
at typical image processing tasks, and whether utilizing the \ac{SIMD}
array yields any performance advantages over doing everything in the
control core. \TODO{noen FPGA-folk som vil skrive om dette?}

%As the assignment was to create an array-based image processor, it
%would be interesting to see how much more efficient our design is, depending on
%the amount of cores the system has.

In this chapter, we will present the results of our benchmarks of two
critical parts of the system: the SD card read performance of the
\ac{SCU} and the image processing performance of LENA.

%The first section will go into detail on the performance aspect of the results. We
%will here look at throughput, total processing time and other aspects related to
%performance, and compare them based on the number of cores we use. In the
%following section, we will have a look at the power consumption. We will measure
%how much power the system requires running with different cores, and how much
%power the system requires to process a single task with different amount of
%cores.

{\sc fancy image here}

\section{Performance}


In this section, there will be data as specified in the disposition. We will run
a program which requires data from neighbors, and see how long time the
different programs take based on the amount of cores. We will then explain
eventual results | for instance would the results most likely point out that we
have low latency, and that it will take a very short amount of time to get data
from neighbors.

There will also be some fancy plots and tables here to show the decrease in time
and/or the increase in speedup.

\section{Power}

In this section, there will be data as specified in the disposition. We will run
a program which requires data from neighbours, and see how much power the
different programs take based on the amount of cores. We will then explain
eventual results. Compared to the other group, we have completely ignored power
consumption, and e.g. our voltage regulators (?) just burn off the additional
power they get, so it's likely a drastic difference here.

There will also be some fancy plots and tables here to show the (incredibly
high) power consumption.

