\section {Introduction}

Design choices: \\
\subsection {Memory}
Since we decided to have separate instruction/data-memories, as well as VGA-ram, and
a dedicated memory for the AVR, we ended up with a total of 5 RAM-chips. The requirements
on the various chips did differ a bit though, we needed 8-bit words for all data, but
24 bit for instructions. Since 24-bit memory was out of production, and 32-bit memory
was too expensive, we opted for using two 16-bit chips, with their address-lines connected
together, and 8 ignored I/O-pins, thus in effect creating a 24-bit memory. Additionally,
as we were unsure about how constant we could get our data-streams, an additional
AVR-memory was added. This was chosen to be the same chip as the VGA-memory.

\subsection {VGA}
While we had a goal of creating our own VGA-controller in the FPGA, we decided to have
some fallback-option, thus, in addition to the neccessary pins and connectors for that
solution, we also added the necessary connectors for the VGA-module that Festina Lente
used last year.

\subsection {Communication}
We planned on using the SD-Card-reader as our main source of data/instructions, however
in the same vein as the design choices for the VGA, we opted to also have a fallback-solution
here, thus we also added USB and RS232 as fallback-solutions for getting data/communicating
with the computer.

