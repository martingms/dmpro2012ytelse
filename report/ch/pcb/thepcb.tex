\section {The PCB}

\begin{figure}[h]
  \centering
  \includegraphics[width=0.8\textwidth]{fig/pcb/pcbwithoutcomp.jpg}
  \caption{The PCB Without Components}
  \label{fig:pcb-without-components}
\end{figure}


When the \ac{PCB} arrived from production, the soldering proved
to be a bit more difficult than originally anticipated.
First, we tested whether
there were any short circuits in the board itself. We used the multimeter to test that
there was no current passing from one layer to another. Then, we started to solder the power supply, one power plane at time, testing the board for short
circuit after each iteration. The table \ref{fig:pcb} shows the observed voltage from each plane.

\begin{table}[h]
  \centering
  \begin{tabularx}{\textwidth}{l l l l}\toprule
    \thx{Test} & \thx{Result} & \thx{Passed} 
    \\ 
	 \midrule
    Power supply 12.0V               &Measured 12.045  & OK  \\	
    Power supply 5.0V               &Measured 4.995  & OK  \\
    Power supply 3.3V                   & Measured 3.286 & OK  \\
    Power supply 2.5V                 & Measured 2.510 & OK \\
    Power supply 1.2V            & Measured 1.240 & OK  \\
    
    \bottomrule
  \end{tabularx}
  \caption{Results of power supply}
  \label{fig:pcb}
\end{table}


After that we started soldering the AVR. Getting the AVR in place was particularly difficult,
but after asking Tufte, we lent a glue stick to put some glue on. The increased friction helped keeping the AVR in place during soldering.

After about a days work we managed to completely destroy a pin on the AVR,
therefore we had to start from scratch on a new board. On this new board we
started with the AVR and \ac{FPGA}, as they were the hardest components to
solder. After each side on the AVR and \ac{FPGA}, we tested for short
circuits. As none were found, we moved on to the power supply, as well as
\acp{JTAG}, FLASH and \acp{LED}. In order to check the \ac{PCB} was working, the
AVR and \ac{FPGA} groups tested the \ac{PCB} board without the capacitors, after
both groups tested that they could connect to the AVR and \ac{FPGA}
respectively, we began to solder the rest part of the \ac{PCB}.

\begin{figure}[h]
  \centering
  \includegraphics[width=0.8\textwidth]{fig/pcb/pcbwithcomp.jpg}
  \caption[The PCB]{The PCB With All The Components.}
  \label{fig:pcb-with-components}
\end{figure}

