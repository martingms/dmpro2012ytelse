\section {Routing}

We spent the entire final week before production working on the routing,
the reason for the time span needed for this was a result of a couple of
things.

\TODO{Expand the following into text}
There are quite a few possible reasons for why we ended up using more time than
the energy group to get our board fully routed:
\begin {itemize}
\item We were the first group to start routing, and thus got to fall into all
  the gotcha-traps that exist for routing, with no-one to walk into them ahead
  of us.
\item We did not think about physical proximity when laying out the pins.
\item We had 5 memory chips, with all the extra connections that implies.
\item We did not set limitations until a few days into the week.
\end {itemize}

Initially we attempted autorouting. This took the better part of half an hour,
even before the constraints were set. This showed us quite a few issues that
needed to be handled, and even more so when we finally got the constraints in.

\TODO{Fill in about the types of errors/warnings we got.}
\begin{itemize}
\item Constraints violations set wrong.
\item Power-plane net-labeled wrong.
\item Clearance constraints violated by Altium.
\item Short-circuiting vias.
\item Overlapping vias.
\item In general, the auto-routing started to produce violations.
\item more\ldots
\end{itemize}
\TODO{Silk-over-silk in USB/Lenna, errors we could ignore.}
\TODO{Scaling of board. (Should go elsewhere or not at all?)}

Initially we started with a board size chosen more or less at random, and picked
something that seemed to fit the components comfortably with room to spare.
After laying out the components on the board, we noticed that the board had
quite a bit of room left over. As this would be a waste of resources, we scaled
the board down in size. We did this by moving the keep-out-borders inwards until
the wasted space was removed, and then used the ``scale-to-fit-components''-tool
in Altium, with the Keepout-border selected.

In the end we remapped some pins to increase physical proximity, and to untangle
the amount of crossing wires. Although care was taken during this process, we
accidentally happened to disconnect one pin in the schematic while reordering.
We did notice this before production, and were able to correct the mistake by
manually routing the connection. Finally, we did some manual routing to reduce
the amount of unneccessary vias.

\subsection {Lenna and Text}
In the top right corner of our board, we put the ``Lenna''-picture. This image
was produced by doing an edge detection pass on the image, then putting it
through the image converter plugin in Altium:
\url{http://wiki.altium.com/display/ADOH/How+to+import+a+graphic+onto+the+PCB+overlay}

We also put our names and the name of the project on the board. This was simply
done with the Text-label tool in Altium's PCB-Designer. As this did not allow
for Norwegian letters, we added lines and circles manually.
