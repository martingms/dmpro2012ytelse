\subsection {Routing}
\section {Routing}

We spent the entire final week before production working on the routing,
the reason for the time span needed for this was a result of a couple of
things.

(TODO: Expand the following into text)
Why we used so much time: (Or, what the other group did right, and we did wrong)
\begin {itemize}
\item We were the first group to start routing, and thus got to fall into all the gotcha-traps that
exist for routing, with no-one to walk into them ahead of us.
\item We didn't think about physical proximity when laying out the pins
\item We had 5 memory chips, with all the extra connections that implies.
\item We didn't set limitations until a few days into the week.
\end {itemize}

Initially we attempted auto-routing, which took the better part of half an hour,
even before the constraints were set, this showed us quite a few issues that needed
to be handled (and even more so when we finally got the constraints in).

TODO: Fill in about the types of errors/warnings we got.
\begin{itemize}
\item Constraints violations set wrong
\item Power-plane net-labeled wrong.
\item Clearance constraints violated by Altium
\item Short-circuiting vias.
\item Overlapping vias.
\item In general, the auto-routing started to produce violations 
\item more...
\end{itemize}
TODO: Silk-over-silk in USB/Lenna, errors we could ignore.
TODO: Scaling of board. (Should go elsewhere or not at all?)
Initially we started with a boardsize chosen more or less at random, just picking something
that seemed to fit the components comfortably with room to spare, after laying out the components
on the board, we noticed that the board had quite a bit of room left over, as this would be a waste
of resources, we scaled the board down in size, by moving the keep-out-borders inwards until the wasted
space was removed, and then using the "scale-to-fit-components"-tool in Altium, with the Keepout-border
selected.

In the end we remapped some pins to increase physical proximity, and to untangle the
amount of crossing wires. Although care was taken during this process, we accidentally
happened to disconnect one pin in the schematic while reordering, we did notice this before
production though, and were able to correct the mistake by manually routing the connection.
Finally we did some manual routing to reduce the amount of unneccessary vias. \\

\subsection {Lenna and Text}
In the top right corner of our board, we put the "Lenna"-picture, this image
was produced by doing an edge-detection-pass on the image, then putting it through
the image-converter-plugin in Altium: \url{http://wiki.altium.com/display/ADOH/How+to+import+a+graphic+onto+the+PCB+overlay}

We also put our names, and the name of the project on the board, this was simply done
with the Text-label tool in Altium's PCB-Designer. This didn't allow for norwegian letters,
but this was fixed easily by adding lines and circles manually (as well as moving a's and e's together
to form æ).

Our plan was to start with a big PCB document to be able to easily place all the components. We placed the components and started routing. A lot of the routing was done by the auto route feature of Altium. At first, we did not set any Design Rules. We found that the default rules did not comply with capabilities listed on Elprints' webpage.\footnote {\url{http://www.elprint.no/products/pcb/capabilities}}. After fixing this, we shrank the board size. The main reason was to reduce production cost, but Tufte also told us that having the board too big could cause short circuits between the power planes.

We then remapped some pins in the schematic to increase physical proximity, and to untangle the amount of crossing wires. Although care was taken during this process, we accidentally managed to disconnect one pin in the schematic while reordering. We did notice this before production, and were able to correct the mistake by manually routing the connection.

In the end, we did 2 days of manual routing to reduce the number of vias. Figure \ref{fig:removingvias-pcb} shows an example where we coupled together several pins to the same via going to the ground layer. Even though we wanted to reduce the number of vias, we went with the default routing in Altium. The amount of manual work could perhaps have been reduced by selecting the "Via Mixer" strategy instead. \TODO{Discussion, not process?}

\begin{figure}[h]
  \centering
  \includegraphics[width=\textwidth]{fig/pcb/pcb_removing_vias.png}
  \caption{Removing vias}
  \label{fig:removingvias-pcb}
\end{figure}

