\section {Process}
This section describes the work and design challenges faced related to the PCB.
\subsection{Memory}
As discussed earlier, as one of our earliest design choices we chose 3 separate memories to allow overlap of
memory accesses. Since the requirements for the data/instruction memories differed in both size and word-width 
we wound up with not only separate, but also different chips for this purpose. The same choice meant we also
needed a separate memory for our \ac{VGA} controller, as that needed to read it's buffer as fast as possible
without interfering with the speed of the rest of the system. This called for a memory that was big enough to
hold atleast a full screen-frame, at 8-bit per pixel (since each pixel is an 8-bit greyscale pixel).

To reduce the possibility of having too slow data-access from the AVR, an extra memory was added to work as
a buffer for the AVR as well. This design choice was made {\em after} ordering, which meant that we had to choose from
the chips we had already ordered to fit this purpose. Since this was intended to carry data intended for the rest
of the system, and as the rest of the system is working with data in 8-bit bytes, we ended up using one of the
extra chips ordered as \ac{VGA} memory for this purpose.

\section {Schematics}

We decided to design the entire PCB in one schematic, because having read
the Festina Lente report, we noticed that they mention that using multiple schematics
might give some interesting issues (which we never got any further into,
as we never tried to use more than one schematic).

A major downside of this approach was the fact that this completely serialized our
work on the Schematic, since we couldn't make concurrent changes to our single document.
This limited the paralellism of our work-flow to one person working on the schematic,
and the other people in the PCB-group doing whatever could be done without touching the schematic.
(Looking up components, making footprints etc). Since the schematic was the biggest amount
of work, this produced a major bottle-neck for the schematic work. HOWEVER, having the overall
control of the entire thing in one document did help to smooth out quite a few issues we met
along the road. (TODO: Fill in a bit of details about pros for one-document vs multi).

The overall layout of the schematic is logically grouped \"geographically\", to
allow for easy reading of the schematic.

\subsection {Buses/Wirelabels}
(TODO: Noe om issues vi fikk med å navngi busser)

TODO: Overordnet forklaring av hvordan

\section {Routing}

We spent the entire final week before production working on the routing,
the reason for the time span needed for this was a result of a couple of
things.

(TODO: Expand the following into text)
Why we used so much time: (Or, what the other group did right, and we did wrong)
\begin {itemize}
\item We were the first group to start routing, and thus got to fall into all the gotcha-traps that
exist for routing, with no-one to walk into them ahead of us.
\item We didn't think about physical proximity when laying out the pins
\item We had 5 memory chips, with all the extra connections that implies.
\item We didn't set limitations until a few days into the week.
\end {itemize}

Initially we attempted auto-routing, which took the better part of half an hour,
even before the constraints were set, this showed us quite a few issues that needed
to be handled (and even more so when we finally got the constraints in).

TODO: Fill in about the types of errors/warnings we got.
\begin{itemize}
\item Constraints violations set wrong
\item Power-plane net-labeled wrong.
\item Clearance constraints violated by Altium
\item Short-circuiting vias.
\item Overlapping vias.
\item In general, the auto-routing started to produce violations 
\item more...
\end{itemize}
TODO: Silk-over-silk in USB/Lenna, errors we could ignore.
TODO: Scaling of board. (Should go elsewhere or not at all?)
Initially we started with a boardsize chosen more or less at random, just picking something
that seemed to fit the components comfortably with room to spare, after laying out the components
on the board, we noticed that the board had quite a bit of room left over, as this would be a waste
of resources, we scaled the board down in size, by moving the keep-out-borders inwards until the wasted
space was removed, and then using the "scale-to-fit-components"-tool in Altium, with the Keepout-border
selected.

In the end we remapped some pins to increase physical proximity, and to untangle the
amount of crossing wires. Although care was taken during this process, we accidentally
happened to disconnect one pin in the schematic while reordering, we did notice this before
production though, and were able to correct the mistake by manually routing the connection.
Finally we did some manual routing to reduce the amount of unneccessary vias. \\

\subsection {Lenna and Text}
In the top right corner of our board, we put the "Lenna"-picture, this image
was produced by doing an edge-detection-pass on the image, then putting it through
the image-converter-plugin in Altium: \url{http://wiki.altium.com/display/ADOH/How+to+import+a+graphic+onto+the+PCB+overlay}

We also put our names, and the name of the project on the board, this was simply done
with the Text-label tool in Altium's PCB-Designer. This didn't allow for norwegian letters,
but this was fixed easily by adding lines and circles manually (as well as moving a's and e's together
to form æ).


\subsection{Soldering}
After that we started soldering the AVR. Getting the AVR in place was particularly difficult,
but after asking Tufte, we lent a glue stick to put some glue on. The increased friction helped keeping the AVR in place during soldering.

After about a days work we managed to completely destroy a pin on the AVR,
therefore we had to start from scratch on a new board. On this new board we
started with the AVR and \ac{FPGA}, as they were the hardest components to
solder. After each side on the AVR and \ac{FPGA}, we tested for short
circuits. As none were found, we moved on to the power supply, as well as
\acp{JTAG}, FLASH and \acp{LED}. In order to check the \ac{PCB} was working, the
AVR and \ac{FPGA} groups tested the \ac{PCB} board without the capacitors, after
both groups tested that they could connect to the AVR and \ac{FPGA}
respectively, we began to solder the rest part of the \ac{PCB}.

